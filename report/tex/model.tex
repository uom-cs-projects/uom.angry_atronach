\chapter{Model}
  A considerable portion of the time allocated to this project was spent on developing a model to represent elements of the learning process. This model underwent a number of changes. In this section I will mostly discuss the final model.

  \section{Description and Justification}
    \todo{
      \item Introduce the underlying thought process in creating the final model: Humans as computers.
      \item Introduce the terminology, then go into detail in the subsections
    }

    \subsection{Learners}
      \todo{
        \item Learners are treated as computers: The assumption is that a human can do everything a computer can \emph{at least}
        \item Acknowledge that human hardware is different from electronic computer hardware: We have strengths and weaknesses that conventional computers don't have
        \item Learners can do visual processing that computers struggle with: Graphs and pictures are a valid way to learn things, often more effective than text.
        \item Learners forget stuff they don't use: JIT learning is a good idea, so that learners immediately start using a skill, hence they don't have to relearn stuff
        \item Learners have very little short term memory (7 things? UI stuff): Resources will have to be designed around avoiding using up this memory: Short, simple sentences are easier to parse as they don't require you to keep track of context.
        \item Learners must have some internals representation of the information that makes up their knowledge: The exact format of this data is not hugely important, although further knowledge of it may indicate easier input formats to use
        \item Learning is the process of translating external information into some internal representation: Most learning tasks are analagous to compilation.
      }

    \subsection{Functions}
      \todo{
        \item Functions are things that people can do
        \item Functions are best defined by their tests
        \item Functions can be extended to include physical functionality, like walking or being able to pick something up: This allows the model to potentially be used to handle skills that aren't purely mental
        \item Some functions may not be possible without other functions: They have dependencies
      }

    \subsection{Implementations}
      \todo{
        \item Actions: A Function that is applied in order to learn an implementation of a function.
        \item Sources: A source of information that provides a working implementation of a function when the necessary actions are applied to it.
        \item implementations can only cover a single weakly connected subgraph of the Functions
      }

    \subsection{Tests}
      \todo{
        \item These are used to determine whether a Learner has gained a particular Function.
        \item The test need to be carefully constructed to avoid making them game-able: Like normal unit tests, they should test edge cases. Many real tests only test normal cases.
      }

    \subsection{Root Functionality}
      \todo{
        \item Perhaps this needs a better name?
        \item Abstracts away the details of the bottom of the graph: There are lots of low level skills we will probably have a lot of trouble working out: e.g. What functionality does a normal baby have? What are the most basic abilities of a human the lead to all other skills being learned (what's built into the hardware?)
      }

  \section{Consequences}
    \todo{
      \item Dependencies define possible orders in which learners can successfully approach functionality
      \item Acyclic graph of dependencies
      \item Talk about specific consequences of specific parts of the model.
      \item Simpler sentences are easier to understand: Compiling and memory issues: Use example sentence from paper.
    }

  \section{Applications}
    \todo{
      \item Summary rather than in-depth analysis
      \item Troubleshooting using backtracking
      \item Study plan generation given goals
      \item Time estimates for work if extra metadata is added
      \item More accurate representation of a learner's actual abilities
    }

  \section{Applying the Model}
    \todo{
      \item A subject area can be refined into smaller and smaller functions
      \item Take an existing resource and treat the abilities gained from that resource as one big function
      \item Try to strip out part of the resource into a seperate function
      \item The major operations: There are primitives, but it's best to capture compound operations that are logically connected as a single operation
    }

    \subsection{Refinement}
      \subsubsection{Serial Split}
      \subsubsection{Parallel Split}
    \subsection{Addition}
      \subsubsection{Extend}
    \subsection{Correction}
      \subsubsection{Inject}
      \subsubsection{Merge}
      \subsubsection{Relink}
        \todo{
          \item unlink and link primitives are available, but it is better to allow many primitive linking operations at once to capture things like mistaken links: The link target is corrected rather than the link's existence.
        }

      \subsubsection{Delete}

  \section{Key Changes}
    \todo{
      \item Started out approaching the problem thinking about information: The idea was to look at text and identify references. After creating my seminar, I realised the problem I was trying to solve had a lot in common with software design: Used a factorial function to explain my ideas
      \item Old idea based around sets of \'equivalent\' representations: Equivalency baffled me, but this was resolved with the concept of a Function.
      \item switched to looking at the problem as that of writing programs for humans to then compile
      \item This swithc solved a number of problems, including the seperation of language from high level functionality: Functions are language independent, implementations rely on language.
      \item Old model used the idea of an AND and OR node, to handle the idea of required dependencies and optional dependencies. The creation of Functions and Implementations more neatly solved this issue: Functions must have all of their deps satisfied, while which implementation is chosen offers choice.
      \item The switch away from studying text based resources meant it was effortless to extend the model to cover videos, images etc. Before I had to talk about the fact that all information sources could eventually be seen as a sequence of symbols.
      \item I had issues deciding how to name concepts: With the idea of Functions having unit tests, naming uniquely is less important: The tests disambiguate things: Note extension, can seperate branches be identified? Can terminology be namespaced?
    }
