\chapter{Software Design}
\label{chapter:design}
  This chapter describes and justifies the final design decisions I made, as well as explaining some of the key changes I made to the design as I learned more. Technical details are left to Chapter \ref{chapter:implementation}: This chapter remains at a higher level, focussing on the model used and the overall architecture of the system.

  \section{What To Build}
    \todo{
      \item The idea is to create a wiki style web application that uses my model
      \item The purpose is to act as a prototype for future tools, and a way to test how practical my model is for people to actually use
    }
  \section{Requirements}
    \todo{
      \item Didn't want to do design up front, so my requirements are necessarily high level. The plan was to develop a core set of functionality, and then try and implement some of the Applicatiosn of the model, refining the design based on feedback
      \item Core functionality includes:
      \item implementing the model in some database
      \item implementing a way to view the data
      \item Implementing a web based UI that allows the primitive operations to be carried out
      \item Implementing the compound operations
      \item Implementing the most basic application: Generating a list of actions to take given a goal or set of goals
      \item Given that the software I managed to make didn't get very far, I didn't get around to implementing any of the other features I had available to work on
    }

  \section{Architecture}
    \todo{
      \item Mention the intended way of structuring the program
      \item Database with an API: The model
      \item Application built on top of that data source
      \item The idea would be to eventually make the data source a service, which would allow third-party applications to interact witht the system: I wanted to come up with a standard for the model to allow external development
      \item Standard MVC architecture for the web app
    }
