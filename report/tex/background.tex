\chapter{Background Knowledge}
\label{chapter:background}
  This chapter details the knowledge required to understand this report. There is also review of various educational practices, which forms the basis for many of my design choices.

  \section{The Learning Process}
    Regardless of how it is implemented, a working learning process needs the following features:

    \begin{itemize}
      \item A way of selecting goals: Somehow we must decide what it is we want to learn.

      \item A way to motivate the student: It is not possible to learn \emph{for} someone. Learning requires students to do a lot of thinking on their own, which will only happen if they are sufficiently motivated.

      \item A method of acquiring new knowledge: This can vary from reading books, to attending lectures, to inventing entirely new knowledge. Note that research can be viewed as a special type of learning.

      \item A method of testing whether the new knowledge was successfully acquired: If there is no measure of success, then there is no way to be certain any progress was made.

      \item A method of dealing with failure: Although it is conceivably possible that a student could never make any mistakes, most students don't have perfect performance. There must be a process to detect the cause of problems and find a solution in any system of practical use.
    \end{itemize}

  \section{Existing Work}
    \todo{
      \item Bloom's Taxonomy
      \item Constructivist Theory
    }

  \section{Review of Current Practices}
    It seems logical to start an attempt to make learning easier with a review of the currently used practices.

    \todo{
      \item Mention the key areas for improvement
      \item Summarise any other key findings
    }

    \subsection{Selecting Goals}
      \todo{
        \item There are a number of choices over what to learn that do not seem to make very much sense: What make everyone study Shakespeare? What about specialist subjects that are not optional? Why is critical thinking not part of the core curriculum? Why are widely useful software engineering skills left until third year, yet group project work is done predominantely in the first and second years?

        \item Selecting goals is difficult: It requires knowledge of the distribution of skills being used in industry. It also requires sufficient structural information about knowledge to distinguish skills from each other and teach them in isolation.

        \item What we really need is data: How is our current knowledge structured, and how many people use individual skills (and in which professions).

        \item Timing is important: JIT learning will lower the amount of knowledge that is forgotten: Use it or lose it! If you learn too soon you won't be using that knowledge. If you're not using the knowledge, you have to put in additional effort memorising for tests.
      }

    \subsection{Student Motivation}
      \todo{
        \item Inappropriately selected goals lead to disillusioned students: They know what they are doing isn't practically useful. Their performance is decreased unless they are exceptional \emph{grinders}.

        \item Motivation seems to be often forgotten. Not much effort is made to show how lessons are of practical value to students. It would be good if every skill could be linked to real examples: This is the essence of \emph{roles}.

        \item Some courses attempt to motivate students through real-life examples. This has mixed results: Sometimes the examples are not realistic, other times they are realistic but inapproriate to the student's goals (maths may be useful for a scientist, but maybe the student is hoping to work in textiles: They must know that maths is useful in their area of interest).
      }

    \subsection{Knowledge Acquisition}
      \todo{
        \item Mention how shallow this review goes (not exhaustive)
        \item There are many methods used to communicate and receive knowledge: Many are modifications or combinations of others so review a few
      }

      \subsubsection{Lectures}
        \todo{
          \item Not very interactive usually owing to size and time restrictions
          \item Inefficient use of a lecturer's time: You could record segments of a lecture and have a computer play it back instead of having a lecturer playback the lecture every year
          \item real-time nature prevents you from taking notes effectively: Either the lecturer must provide notes or the student must split their attention between recording information and listening
          \item It can be hard to judge what is important, making it even harder to take good notes
          \item Note-taking is replicated work between students
          \item Apparently, people think taking notes will cause people to remember the information: Far more effective would be doing tasks that require that information as it is less boring, realistic practice and avoids memorisation of information that never gets used. JIT learning.
          \item Notes are incomplete: Can't correct errors using them.
          \item If you don't keep up with a lecture, you're screwed: You can't pause and replay parts of a lecture.
        }

      \subsubsection{Labs}
        \todo{
          \item Tend to identify misunderstandings, as real work can't be accomplished with a flawed understanding: Debugging
          \item Problems can be resolved because staff and other students are usually able to help: People can read the instructions and material at their own pace
        }

      \subsubsection{Textbooks}
        \todo{
          \item Flexible: No time pressure
          \item Often there is an attempt to describe which chapters are necessary and which are optional: Clearly there is dependency between concepts
          \item Writing styles vary: Some books are very easy to follow, others are hard: Why is this?
          \item Within a textbook, things are neatly structured and consistent
          \item Things break down when you start asking what knowledge you need going into the book
          \item There's usually either overlap between textbooks, or gaps that aren't easy to fill
          \item Textbooks are not exhaustive in their pre-requisites: Sometimes you may get a book that mentions GCSEs or A levels, but given how fuzzy those qualifications are for determining exact knowledge, there will be overlap, or missing knowledge
          \item Many courses do not use textbooks as there are none that suitably cover the relevant material: Lecturers make their own custom selections of topics
        }

    \subsection{Testing}
      \todo{
        \item Group testing is meaningless outside of that group (can't isolate individual performance, group will be disbanded soon, eliminating utility of test)

        \item Lots of tests collapse information into a single score: Detail is lost, utility is compromised. A single percentage score cannot tell you whether a student undestands A, B, C, AB, AC, BC, or nothing completely. Only a 100\% or 0\% score tells you anything concrete about their actual knowledge. Testing multiple variables with one score makes no sense! Tests need to be designed with utility in mind: Who is going to use the test score and what does it reliably tell them?

        \item Lots of tests are game-able: Answers can be memorized and there are numerous examples of lots of time being dedicated to preparing for specific exam techniques and model answers in my education. The tests can't reliably seperate those who fully understand the questions, and those who \emph{grind}.

        \item Many tests only test a portion of a course. Why put something on a course if you are not going to check that it has been learened? That's much like adding features to a program but never writing tests for them. Who knows if your work was worth it?
      }

    \subsection{Handling Failed Tests}
      \todo{
        \item Educational institutions run on very fixed schedules: Problems have a limited window in which to be fixed before no correction is made, or high costs are incurred.

        \item Review is far less common than it should be! Kind of crippled because of the lack of accuracy of many tests: What is the actual problem? Unclear.

        \item Failure to adapt teaching schedule to match students: If you're stuck on something, regular lessons will continue without you, and it's worthless going to something that relies on the problem area. Catch up gets more expensive the longer it is left.
      }
