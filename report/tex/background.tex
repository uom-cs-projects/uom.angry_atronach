\chapter{Background Knowledge}
\label{chapter:background}
  This chapter details the knowledge required to understand this report. There is also review of various educational practices, which forms the basis for many of my design choices.

  \section{Existing Work}
  \section{Review of Current Practices}
    It seems logical to start an attempt to make learning easier with a review of the currently used practices. As you will see below, there are a number of suboptimal practices currently in use which could be improved with the use of out current level of technology.
    
    \todo{
      \item Mention the key areas for improvement
      \item Summarise any other key findings
    }

    \subsection{The Learning Process}
      Regardless of how it is implemented, a working learning process needs the following features:

      \begin{itemize}
        \item A way of selecting goals: Somehow we must decide what it is we want to learn.

        \item A way to motivate the student: It is not possible to learn \emph{for} someone. Learning requires students to do a lot of thinking on their own, which will only happen if they are sufficiently motivated.

        \item A method of acquiring new knowledge: This can vary from reading books, to attending lectures, to inventing entirely new knowledge. Note that research can be viewed as a special type of learning.

        \item A method of testing whether the new knowledge was successfully acquired: If there is no measure of success, then there is no way to be certain any progress was made.

        \item A method of dealing with failure: Although it is conceivably possible that a student could never make any mistakes, most students don't have perfect performance. There must be a process to detect the cause of problems and find a solution in any system of practical use.
      \end{itemize}

    \subsection{Selecting Goals}
    \subsection{Student Motivation}
    \subsection{Knowledge Acquisition}
    \subsection{Testing}
    \subsection{Handling Failed Tests}
